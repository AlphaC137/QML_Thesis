\chapter{Literature Review: Quantum-enhanced Machine Learning}\label{sec:qml}

Classical machine learning takes classical data as input and learns from it using classical algorithms executed on classical computers - this will be referred to as C/C (classical data with classical algorithm). One enters the field of quantum machine learning when either quantum data or quantum algorithms are combined with ideas from classical machine learning. Thus, quantum machine learning can be subdivided into three different subfields: 1) C/Q - classical data with quantum algorithm, 2) Q/C - quantum data with classical algorithm and 3) Q/Q - quantum data with quantum algorithm.

\emph{Quantum data} includes any data describing a quantum mechanical system such as e.g. the Hamiltonian of a system or the state vector of a certain quantum state. A \emph{quantum algorithm} is any algorithm that can only be executed on a quantum computer. 

C/Q is the topic of this thesis

Analogously, Q/C is the subfield where quantum data is being processed using a classical machine learning algorithm. Examples for Q/C:

used genetic algorithms to reduce digi-
tal and experimental errors in quantum gates

A tantamount task is then to
find a model (a.k.a. effective) Hamiltonian of the system
and to determine properties of the present noise sources.
By computing likelihood functions in an adaptation of
Bayesian inference, Wiebe et al. found that quan-
tum Hamiltonian learning can be performed using realis-
tic resources such as depolarizing noise.

Q/Q would include learning a model Hamiltonian of a system implemented in a lab using a quantum learning algorithm.

\subsection{Quantum State Preparation Algorithms}
\label{subsubsec:quantumstatepreparation}

\emph{Quantum state preparation} is the process of preparing a quantum state that accurately represents a vector containing classical (normalized) data. There are two fundamentally different ways of preparing a quantum state representing the classical example vector v:

\begin{equation}
v = \begin{pmatrix}0.6 \\ 0.4 \end{pmatrix}
\end{equation}

\subsubsection{Encoding classical data into qubits}
\label{subsubsec:classicaldataqubits}
%speed-up not very clear since the \# of qubits increases linearly with the \# of classical bits
The most straightforward type of quantum state preparation does not make use of quantum superpositions but only uses the definite \0 or \1 states to store information in a multi qubit system as outlined in the example below.

Multiply vector v by ten such that probabilities can easily be represented in binary,
\begin{equation}
\begin{pmatrix}
 \textcolor{blue}{0.6} \\ 
 \textcolor{emerald}{0.4}
 \end{pmatrix}*10 \rightarrow \begin{pmatrix}
 \textcolor{blue}{6} \\ 
 \textcolor{emerald}{4}
 \end{pmatrix}
\end{equation}
 Convert each entry to binary,
 \begin{equation}
 \begin{pmatrix}
 \textcolor{blue}{6} \\ 
 \textcolor{emerald}{4}
 \end{pmatrix} \rightarrow \begin{pmatrix}
 \textcolor{blue}{0110} \\ 
 \textcolor{emerald}{0100}
 \end{pmatrix}
 \end{equation}

 Rewrite 2-D vector as 1-D bit string,
 \begin{equation}
 \begin{pmatrix}
 \textcolor{blue}{0110} \\ 
 \textcolor{emerald}{0100}
 \end{pmatrix} \rightarrow n=\textcolor{blue}{0110}\textcolor{emerald}{0100}
\end{equation}
For $n$ bits initialize $n$ qubits \& apply X gate to respective qubits:
\begin{equation}
n=\textcolor{blue}{0110}\textcolor{emerald}{0100}  \rightarrow \ket{n} = \ket{\textcolor{blue}{0110}\textcolor{emerald}{0100}}
\end{equation}
%\textbf{Only slight speed up possible}

When encoding classical data into qubit states, a $k$-dimensional probability vector requires $4k$ classical bits which are encoded one-to-one into $4k$ qubits. Thus, the number of qubits increases linearly with the size of the classical data vector. Due to this one-to-one correspondence between classical bits and qubits there is no improvement with respect to data storage. Only slight (up to quadratic) speed-ups are possible through clever quantum algorithm design.


Qubit-based quantum state preparation becomes slightly more complicated when aiming to achieve an equal superposition of $l$ binary patterns of the form

\begin{equation}
\ket{M} = \frac{1}{\sqrt{l}}\sum^l_{j=1} \ket{l^j}
\end{equation}

which is a requirement for the later used qubit-encoded kNN quantum algorithm by \citeA{Schuld2014}. \citeA{Trugenberger2001} describe a quantum routine that can efficiently prepare such a state as will be explained in detail below.

Describe
Trugenberger et al.
here


\subsubsection{Encoding classical data into amplitudes}
\label{subsubsec:classicaldataamplitudes}

A more sophisticated way of representing the classical vector $v$ as a quantum state makes use of the large number of available amplitudes in a multi qubit system. The general idea is demonstrated in Equ.~\ref{equ:amplitudedata} below.

\begin{equation}
\label{equ:amplitudedata}
\begin{pmatrix}
 \textcolor{blue}{0.6} \\ 
 \textcolor{emerald}{0.4}
 \end{pmatrix} \quad \rightarrow \quad \ket{n} = \sqrt{\textcolor{blue}{0.6}}\ket{0}+\sqrt{\textcolor{emerald}{0.4}}\ket{1}
\end{equation}

Using amplitude-based quantum state preparation, a $k$-dimensional probability vector is encoded into only $log_{2}(k)$ qubits since the number of amplitudes grows exponentially with the number of qubits. This type of quantum data storage makes exponential compression of classical data possible. Since applying a quantum gate acts on all amplitudes in the superposition at once there is the possibility of exponential speed-ups in quantum algorithms compared to their classical counterparts. Amplitude-encoding requires much a smaller number of qubits that grow logarithmically with the size of the classical data vector compared to the one-to-one correspondence in qubit-encoded state preparation. However, initializing an arbitrary amplitude distribution is still an active field of research and requires the implementation of non-trivial quantum algorithms.

For the case when the classical data vectors represent discrete probability distributions which are efficiently integrable on a classical computer, \citeA{Grover2002} developed a quantum routine to initialize the corresponding amplitude distribution.
%The main idea in their algorithm is subdividing the respective probability distribution and encoding the probabilty
%VERY COMPLICATED AND MOST GENERAL
Additionally, \citeA{soklakov2006efficient} proposed a quantum algorithm for the the more general case that also includes classical data vectors representing non-efficiently integrable probability distributions.

\subsection{Quantum k-nearest Neighbour Algorithm}
\label{subsubsec:quantumknearestneighbour}

The quantum distance-weighted kNN algorithm outlined in this section was proposed by \citeA{Schuld2014} and is based on classical data being encoded into qubits rather than amplitudes. The first step is to prepare an equal superposition $\ket{T}$ over $N$ training vectors $\vec{v}$ of length $n$ with binary entries $v_1,v_2,...v_n$ each assigned to a class $c$ as follows,

\begin{equation}
\ket{T} = \frac{1}{\sqrt{N}}\sum_p^N \ket{v_1^p,v_2^p,...v_n^p;c^p}
\end{equation}

The unknown vector $\vec{x}$ of length $n$ and binary entries $x_1,x_2,...x_n$ needs to be classified and is added to the training superposition resulting in the initial state $\ket{\psi_0}$:

\begin{equation}
\ket{\psi_0} = \frac{1}{\sqrt{N}}\sum_p^N \ket{x_1^p,x_2^p,...x_n^p;v_1^p,v_2^p,...v_n^p;c^p}
\end{equation}

An ancilla qubit initially in state \0 is added to the state such that the superposition is now described by,

\begin{equation}
\ket{\psi_1} = \frac{1}{\sqrt{N}}\sum_p^N \ket{x_1^p,x_2^p,...x_n^p;v_1^p,v_2^p,...v_n^p;c^p} \otimes \ket{0}
\end{equation}

The state now consists of four registers: 1) input($x$) register, 2) training($v$) register, 3) class($c$) register and 4) ancilla($\ket{0}$) register. Next, the ancilla register is put into an equal superposition by applying an H gate to it,

\begin{align}
\ket{\psi_2} &= \frac{1}{\sqrt{N}}\sum_p^N \ket{x_1^p,x_2^p,...x_n^p;v_1^p,v_2^p,...v_n^p;c^p} \otimes H\ket{0}\notag\\
&= \frac{1}{\sqrt{N}}\sum_p^N \ket{x_1^p,x_2^p,...x_n^p;v_1^p,v_2^p,...v_n^p;c^p} \otimes \frac{(\ket{0}+\ket{1})}{\sqrt{2}}
\end{align}

The main step in any kNN algorithm is calculating some measure of distance between each training vector $\vec{v}$ and the input vector $\vec{x}$ which in this quantum algorithm is taken to be the Hamming distance as defined in the box below.

\begin{redbox}
\textbf{Hamming distance (CITATION?)}\\
\newline
Hamming distance (HD) is the number of differing characters when comparing two equally long binary patterns $p_0$ and $p_1$.\\
\newline
Example:\\
$p_0 = \quad\textcolor{red}{0}\quad\textcolor{green}{0\quad1}$\\
$p_1 = \quad\textcolor{red}{1}\quad\textcolor{green}{0\quad1}$\\
$--------$\\
HD $= 1+0+0 = 1$
\end{redbox}

Given quantum state $\ket{\psi_2}$ the HD between the input and each training register can be calculated by applying CNOT($x_s,v_s$) gates to all qubits in the first and second register using the input vector qubits ($x_s$) as controls and the training vector qubits ($v_s$) as targets. This will change the qubits in the second register according to the rules specified in Equ.~\ref{equ:distancerules}. 

\begin{equation}
\label{equ:distancerules}
d^p_k =
    \begin{cases}
      0, & \text{if}\ \ket{v_k^p} = \ket{x_k} \\
      1, & \text{otherwise}
    \end{cases}
\end{equation}

The sum of the qubits in the second register now represents the total HD between each training register and the input. Applying an X gate to each qubit in the second register reverses the HD such that small HDs become large and vice versa. This is crucial since training vectors close to the input should be get larger weights than more distant ones. The quantum state is now given by,

\begin{align}
\ket{\psi_3} &= \prod_{s=1}^n X(x_s)CNOT(x_s,v_s)\ket{\psi_2}\notag\\
&= \frac{1}{\sqrt{N}}\sum_p^N \ket{x_1^p,x_2^p,...x_n^p;d_1^p,d_2^p,...d_n^p;c^p} \otimes \frac{(\ket{0}+\ket{1})}{\sqrt{2}}
\end{align}

By applying the unitary operator $U$ defined by,

\begin{equation}
\label{equ:sumoperator}
U = e^{-i\frac{\pi}{2n}K}
\end{equation}

where

\begin{equation}
\label{equ:sumoperator}
K = \mathbb{1} \otimes \sum_k (\frac{\sigma_z+1}{2})_{d_k} \otimes \mathbb{1} \otimes (\sigma_z)_c
\end{equation}

the sum over the second register is computed. As a result, the total reverse HD, denoted $d_H(\vec{x},\vec{v}^p)$, between the $p$th training vector $\vec{v}^p$ and the input vector $\vec{x}$ is written into the amplitude of the p$th$ term in the superposition. The ancilla register is now separating the superposition into two terms due to a sign difference in the amplitudes (negative sign when ancilla is \1). The result is given by, 

\begin{align}
\ket{\psi_4} &= U\ket{\psi_3}\notag\\
&= \frac{1}{\sqrt{2N}}\sum_p^N e^{i\frac{\pi}{2n}d_H(\vec{x},\vec{v}^p)} \ket{x_1^p,x_2^p,...x_n^p;d_1^p,d_2^p,...d_n^p;c^p;0} \notag\\
&\quad\quad\quad\quad\quad\quad + e^{-i\frac{\pi}{2n}d_H(\vec{x},\vec{v}^p)} \ket{x_1^p,x_2^p,...x_n^p;d_1^p,d_2^p,...d_n^p;c^p;1}
\end{align}

Applying an H gate to the ancilla register transfers the $d_H(\vec{x},\vec{v}^p)$ from the phases into the amplitudes such that the new quantum state is described by,

\begin{align}
\label{equ:beforecm}
\ket{\psi_5} &= (\mathbb{1} \otimes \mathbb{1} \otimes \mathbb{1} \otimes H)\ket{\psi_4}\notag\\
&= \frac{1}{\sqrt{N}}\sum_p^N cos\big[\frac{\pi}{2n}d_H(\vec{x},\vec{v}^p)\big] \ket{x_1^p,x_2^p,...x_n^p;d_1^p,d_2^p,...d_n^p;c^p;0} \notag\\
&\quad\quad\quad\quad\quad\quad + sin\big[\frac{\pi}{2n}d_H(\vec{x},\vec{v}^p)\big] \ket{x_1^p,x_2^p,...x_n^p;d_1^p,d_2^p,...d_n^p;c^p;1}
\end{align}

At this point the ancilla qubit is measured along the standard basis and all previous steps have to be repeated until the ancilla is measured in the \0 state. Since it is conditioned on a particular outcome, this type of measurement is called \emph{conditional measurement} (CM). The probability of a successful CM is given by the square of the absolute value of the amplitude and is dependent on the average distance between all training vectors and the input vector:

\begin{equation}
Prob(\ket{a} = \ket{0}) = \sum_p^N cos^2\big[\frac{\pi}{2n}d_H(\vec{x},\vec{v}^p)\big]
\end{equation}

When rewriting Equ.~\ref{equ:beforecm} into the following form,

\begin{align}
\label{equ:beforecm2}
\ket{\psi_5} &= \frac{1}{\sqrt{N}}\sum_{c=1}^d \ket{c} \otimes \sum_{l \in c} cos\big[\frac{\pi}{2n}d_H(\vec{x},\vec{v}^l)\big] \ket{x_1^l,x_2^l,...x_n^l;d_1^l,d_2^l,...d_n^l;0} \notag\\
&\quad\quad\quad\quad\quad\quad\quad\quad\quad\quad + sin\big[\frac{\pi}{2n}d_H(\vec{x},\vec{v}^l)\big] \ket{x_1^l,x_2^l,...x_n^l;d_1^l,d_2^l,...d_n^l;1}
\end{align}

where $l$ runs over all vectors $\vec{v}$ of a particular class c.

Finally, to classify the input vector $\vec{x}$ the class register is measured along the standard basis. The probability of measuring a specific class c is then given by the following expression:

\begin{equation}
\label{equ:classprobs}
Prob(c) = \frac{1}{NProb(\ket{a} = \ket{0})} \sum_{l \in c} cos^2\big[\frac{\pi}{2n}d_H(\vec{x},\vec{v}^l)\big]
\end{equation}

From Equ.~\ref{equ:beforecm2} and~\ref{equ:classprobs} it is evident that the probability of measuring a certain class $c$ is dependent on the average total reverse HD between all training vectors belonging to class $c$ and the input vector $\vec{x}$. Since the total reverse HD represent distance-dependent weights it gets clear why this is the quantum equivalent to a classical distance-weighted kNN algorithm.

In order to obtain a full picture of the probability distribution over the different classes a sufficient number of copies of $\ket{\psi_5}$ need to be prepared and the class qubit needs to be measured after a successful CM on the ancilla qubit.

\begin{redbox}
\textbf{Algorithmic complexity}\\
\newline
According to \citeA{Schuld2014}, the preparation of the superposition has a complexity of $\mathcal{O}(Pn)$ where $P$ is the number of training vectors and $n$ is the length of the feature vectors. The algorithm has to be repeated $T$ times in order to get a statistically precise picture of the results. Hence, the total quantum kNN algorithm has an algorithmic complexity of $\mathcal{O}(TPn)$. 
\end{redbox}

